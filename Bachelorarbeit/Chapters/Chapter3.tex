% Chapter Template

\chapter{Mapping} % Main chapter title

\label{Chapter3} % Change X to a consecutive number; for referencing this chapter elsewhere, use \ref{ChapterX}

%----------------------------------------------------------------------------------------
%	SECTION 1
%----------------------------------------------------------------------------------------

\section{Parallelen und Unterschiede}
Hallo
Im nun folgenden Kapitel wollen wir erläutern in wie fern sich BPMN auf ACP mappen lässt. Hierzu wollen wir zunächst einige parallelen und Unterschiede betrachten. \\
Da sowohl die Moddelierungssprache BPMN als auch die Prozessalgebra ACP genutzt werden um auf unterschliedliche Art Prozesse darzustellen, finden sich zwischen den beiden viele Parallelen. Die von uns betrachtete Teilmenge von BPMN bietet sehr gute Möglichkeiten ein Mapping durchzuführen. Einige wichtige Aspekte wurden bereits im Abschnitt 2.2 genannt. Die Tasks aus BPMN können als Aktivitäten in ACP angesehen werden. Beispielsweise könnte also der Prozess aus Abbildung 2.1 beschrieben werden durch $P:=AnmeldedatenEingeben.$ Die Konkatenation von unterschiedlichen Tasks durch die sequence flows können in ACP auch einfach durch den *-Operator dargestellt werden. Auch die Aufspaltung der Diagramme durch die Gateways kann in ACP leicht umgesetzt werden. Der +-Operator und der ||-Operator bieten ein Äquivalent zu den Exclusiven und Parallelen Gateways. Ebenso leicht lassen sich auf Sub-Prozesse umsetzen. Betrachten wir das Beispiel aus Abbildung x.xx, so können wir auch in ACP eine dazu Äquivaltene Menge an Prozessen verfassen. Diese wäre wie folgt definiert \textbf{"TODO"}.\\
Die ersten Komplikationen treten auf bei dem betrachten von Events. In ACP gibt es keine Möglichkeit den Prozess aufzuhalten bis ein Ereignis auftritt. In der hier verwendeten Teilmenge, kann diese Problematik allerdings recht leicht umgangen werden. Einige Objekte wie pools und lanes werden von uns nicht verwenden. Zusätzlich werden auch werfende Events nicht genauer betrachtet. Dadurch ist es ohne größere Konsequenzen möglich, Events genauso wie Tasks zu behandeln und auf die Aktivitäten zu mappen. Da keine werfenden Events auftreten, können die fangenden Events nur von außerhalb des Prozesses getriggert werden. Dadurch ist es möglich das Event durch eine Aktivität zu beschreiben welche Terminiert, sobald das Ereignis auftritt. Startevents und Endevents können genauso durch aktivitäten beschrieben werden wie intermediate Events. \\
Eingige Beispiele wie das Mapping durchgeführt werden könnte sind in Abbildung x.xx zu finden. \textbf{"TODO"}
In der Abbildung ist leicht zu erkennen, dass Formeln in ACP sehr schnell unübersichtlich werden. Desshalb wollen wir durch das Mapping den Prozess in BPMN in eine Menge von Prozessen und Subprozessen in ACP aufteilen. In Kapitel 4 wird sich zusätzlich zeigen, dass das aufspalten eines Prozesses in mehrere Subprozesse die Implementierung des Mappings deutlich vereinfacht.\\

\section {Aufspaltung in Subprozesse}
Im folgenden Abschnitt wollen wir nun genauer betrachten an welchen Stellen ein Prozess am Sinnvollsten aufgespalten werden kann. Zusätzlich müssen noch einige Annahmen und Einschränkungen über die beiden Arten von Gateways gemacht werden.\\
Im folgenden werden Tasks und Events gleichgesetzt behandelt und es wird nicht mehr zwischen den beiden unterschieden. Beide Elemente werden nur als Aktivitäten benannt. Folgt auf eine gewöhnlich Aktivität genau eine weitere Aktivität, so wird in ACP kein Subprozess erstellt. Es wird eine einfache Konkatenation der beiden Aktivitäten dem Prozess hinzugefügt. Tritt also kein Gateway in dem Prozessdiagramm auf, so wird der Prozess in ACP auch nicht aufgespalten. Das Mapping verhält sich also im Fall der ersten Zeile so wie in Abbildung x.xx.\\
Für alle weiteren Beispiele aus der Abbildung, werden allerdings Subprozesse erstellt. Die Abbildungs zeigt also nicht die sleben Lösungen wie jene, die von unserem Programm erstellt werden sollen. Hier ist zu beobachten, dass mehrere korrekte Lösungen für das Mapping von BPMN nach ACP existieren.\\
Ein Subprozess wird in unserem Fall für jeden ausgehenden sequence flow von jedem im Diagramm vorkommenden Gateway erstellt. Zuerst wollen wir das Exclusive Gateway näher betrachten. In Abbildung x ist in einem Beispiel dargestellt wie sich das Mapping verhalten soll. In der dritten Spalte ist eine weite korrekte Lösung vorgezeigt. Diese ist dazu da um zu zeigen wie sich unser Programm nicht verhalten soll. Jeder Prozess listet alle Aktivitäten auf bis das er auf ein Gateway stößt. An dem Gateway selbst wird ein neuer Prozess für jeden ausgehenden sequence Flow erstellt und im Anschluss nicht mehr weitergearbeitet. P0 arbeitet also alle Aktivitäten bis zum Gateway ab, verweist dann auf die Prozesse P1 und P2 und terminiert im Anschluss. Es ist zu sehen, dass für jeden ausgehenden seqeunce Flow ein neuer Subprozess erstellt wurde.\\ 
Ebenso ist an dem zusammenführenden Gateway zu kennen, dass auch hier für den ausgehenden Sequenceflow ein neuer Subprozess erstellt wurde. Hier kann nicht einfach eine Konkatenation verwendet werden. Zum einen ist der Abbildung zu sehen, dass es schnell zu unübersichtlichkeit führen kann, da sowohl in P1 als auch in P2 enthalten sein muss wie sich der Prozess nach dem Gateway verhält. Zum anderen hilft das erstellen von Subprozessen bei der Umsetzung von Rückkopplungen und Rekusrionen.\\
In Abbildung x.xx ist ein weiteres Beispiel zu sehen. Hier wird schnell deutlich, dass es nützlich ist auch für das zusammenführende Gateway zu beginn des Prozesses einen Subprozess zu erstellen, damit der Prozess PX auf den bereits vorhandenen Prozess P1 verweisen kann. Außerdem hilft dieser Ansatz bei komplexeren Beispielen mit komplizierteren verzweigungen die Übersichtlichkeit und lesbarkeit der Formel aufrecht zu erhalten.\\

Bei den Parallelen Gateways müssen wir zunächst einige einschränkungen machen. Einige verzweigungen die bei den Exclusiven Gateways genutzt werden, dürfen bei den Parallelen Gateways nicht vorkommen.\\


Hier würde ich gerne die Annahme machen, dass bei parallelen Gateways gleich viele Stränge das zusammenfphrende Gateway treffen wie im aufspaltenden erstellt werden. Außerdem soll kein Strang den bereich zwischen den beiden Gateways verlassen. Der Raum ziwschen einen aufspaltenden und zusammenfphrenden Parallelen Gateway soll also wie geschlossen wirken. Kein Strang kann von außen hinein und keiner kann von innen nach außen gelangen. Sie müssen über die Gateways gehen. Das erleichtert die Programmierung und zum anderen macht es glaube ich auch von der Semantik her mehr Sinn. Hier bräuchte ich mal deine Meinung. An sich ist es nicht komplett undenkbar, dass ein Prozess zum Beispiel, wenn er an einer gewissen stelle angekommen ist zwar immernoch weiter machen soll aber gleichzeitig auch schon wieder von vorne anfängt. Das wäre mit meinen Einschränkungen nicht erlaubt. Ich habe es leider schon so programmiert und es wären halt ein paar verschweendete Stunden wenn wir sagen, dass wir diese Einschränkung nicht machen. Außerdem hab ich keine Ahnung wie man das mit den Parallelen Gateways dann Programmieren soll :)
Deine Meinung wäre hier sehr willkommen...






