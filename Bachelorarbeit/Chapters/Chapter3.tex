% Chapter Template

\chapter{Mapping} % Main chapter title

\label{Chapter3} % Change X to a consecutive number; for referencing this chapter elsewhere, use \ref{ChapterX}

%----------------------------------------------------------------------------------------
%	SECTION 1
%----------------------------------------------------------------------------------------

\section{Parallelen und Unterschiede}

Im nun folgenden Kapitel wollen wir erläutern in wie fern sich BPMN auf ACP mappen lässt. Hierzu wollen wir zunächst einige parallelen und Unterschiede betrachten. \\
Da sowohl die Moddelierungssprache BPMN als auch die Prozessalgebra ACP genutzt werden um auf unterschliedliche Art Prozesse darzustellen, finden sich zwischen den beiden viele Parallelen. Die von uns betrachtete Teilmenge von BPMN bietet sehr gute Möglichkeiten ein Mapping durchzuführen. Einige wichtige Aspekte wurden bereits im Abschnitt 2.2 genannt. Die Tasks aus BPMN können als Aktivitäten in ACP angesehen werden. Beispielsweise könnte also der Prozess aus Abbildung 2.1 beschrieben werden durch $P:=AnmeldedatenEingeben.$ Die Konkatenation von unterschiedlichen Tasks durch die sequence flows können in ACP auch einfach durch den *-Operator dargestellt werden. Auch die Aufspaltung der Diagramme durch die Gateways kann in ACP leicht umgesetzt werden. Der +-Operator und der ||-Operator bieten ein Äquivalent zu den Exclusiven und Parallelen Gateways. Ebenso leicht lassen sich auf Sub-Prozesse umsetzen. Betrachten wir das Beispiel aus Abbildung x.xx, so können wir auch in ACP eine dazu Äquivaltene Menge an Prozessen verfassen. Diese wäre wie folgt definiert \textbf{"TODO"}.\\
Die ersten Komplikationen treten auf bei dem betrachten von Events. In ACP gibt es keine Möglichkeit den Prozess aufzuhalten bis ein Ereignis auftritt. In der hier verwendeten Teilmenge, kann diese Problematik allerdings recht leicht umgangen werden. Einige Objekte wie pools und lanes werden von uns nicht verwenden. Zusätzlich werden auch werfende Events nicht genauer betrachtet. Dadurch ist es ohne größere Konsequenzen möglich, Events genauso wie Tasks zu behandeln und auf die Aktivitäten zu mappen. Da keine werfenden Events auftreten, können die fangenden Events nur von außerhalb des Prozesses getriggert werden. Dadurch ist es möglich das Event durch eine Aktivität zu beschreiben welche Terminiert, sobald das Ereignis auftritt. Startevents und Endevents können genauso durch aktivitäten beschrieben werden wie intermediate Events. \\
Eingige Beispiele wie das Mapping durchgeführt werden könnte sind in Abbildung x.xx zu finden. \textbf{"TODO"}
In der Abbildung ist leicht zu erkennen, dass Formeln in ACP sehr schnell unübersichtlich werden. Desshalb wollen wir durch das Mapping den Prozess in BPMN in eine Menge von Prozessen und Subprozessen in ACP aufteilen. In Kapitel 4 wird sich zusätzlich zeigen, dass das aufspalten eines Prozesses in mehrere Subprozesse die Implementierung des Mappings deutlich vereinfacht.\\

\section {Aufspaltung in Subprozesse}
Im folgenden Abschnitt wollen wir nun genauer betrachten an welchen Stellen ein Prozess am Sinnvollsten aufgespalten werden kann. Zusätzlich müssen noch einige Annahmen und Einschränkungen über die beiden Arten von Gateways gemacht werden.\\
Im folgenden werden Tasks und Events gleichgesetzt behandelt und es wird nicht mehr zwischen den beiden unterschieden. Beide Elemente werden nur als Aktivitäten benannt. Folgt auf eine gewöhnlich Aktivität genau eine weitere Aktivität, so wird in ACP kein Subprozess erstellt. Es wird eine einfache Konkatenation der beiden Aktivitäten dem Prozess hinzugefügt. Tritt also kein Gateway in dem Prozessdiagramm auf, so wird der Prozess in ACP nicht aufgespalten. Das Mapping verhält sich also im Fall der ersten Zeile so wie in Abbildung x.xx.\\
Nun müssen wir zwischen den beiden unterschiedlichen Gateways unterscheiden. Zunächst wollen wir den Fall beschreiben, dass ein Exclusives Gateway auftritt.